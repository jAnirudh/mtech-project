\chapter{Introduction}

The use of mathematical models for modelling the behaviour of physical systems has prevailed for as long as one can remember. For typical fluid flow problems, the equations obtained using the Continuum Approximation are most widely used; the yardstick for rigid bodies are the Newton's Laws of Motion. In the case of Isotropic Newtonian fluids, the flow equations boil down to the popular form as given by Navier and Stokes. More often than not, these equations and the associated boundary conditions describing the physical systems of interest, are intractable and do not produce any closed form mathematical expression; thus, one resorts to approximating the variables associated with the system by means of Numerical Methods. Further, the physical systems this work aims to tackle involve solid-solid as well as solid-fluid interactions along with fluid flow; such problems are broadly classified as ``Fluid Structure Interaction (FSI)'' problems. FSI situations are extensively encountered in a number of engineering scenarios spanning a large number of domains and, hence, the ability to accurately capture all three types of interactions is paramount.

A plethora of Numerical Methods are available to model fluid flow; most widely used are those of the Finite Difference Method (FDM), the Finite Volume Method (FVM) and Finite Element Method (FEM). However, these popular methods build on an Eulerian framework and require the problem domain to be discretized explicitly into a grid. In contrast, Particle Methods (also called as Grid/Mesh Free methods) work with a Lagrangian framework and do not demand for domain discretization via a grid. The Smooth Particle Hydrodynamics (SPH) method is one such particle method.

To resolve the forces acting on interacting solid bodies, the Discrete Element Method (DEM) employing various contact models (Soft Contact Models and Hard Contact Models) have been successfully used. Hard Contact Models take into consideration the Coefficient of Restitution of the interacting bodies to estimate the collision force. However for large number of situations, experimental computations of the coefficient of restitution is often tedious, and at times practically untenable. On the other hand, Soft Contact Models allow the interacting bodies to ``merge'' into one another, and using the amount of overlap as the displacement to a spring and damper system, estimate the Collision Forces.

After obtaining the Collision forces, Newton's Laws of Motion are solved to obtain the positions, (linear as well as angular) velocities and accelerations of system of solids thus ending one iteration of the solid-solid interaction. Thereafter, by means of a coupling mechanism, these details are communicated to the fluid, wherein appropriate source terms are incorporated to accommodate the effect of collision. Subsequently, SPH takes over to obtain flow properties. This completes one iteration of the integrated model. The coupling mechanism and the flow field solution obtained using SPH is beyond the scope of this work; the major concentration is on implementation of DEM based Collision models in PySPH. However, for the sake of understanding both the design philosophies of the PySPH tool and the SPH numerical method itself, the SPH method is discussed in a (sufficiently) brief introductory capacity.

The remainder of the report is laid out as follows: Chapters 2 and 3 discuss the details of the SPH and DEM numerical methods and the PySPH framework respectively. Chapter 4 describes a DEM method inspired Proof of Concept (PoC) collision model, its implementation in PySPH and a study performed to validate the model; the chapter is closed off with qualitative results to demonstrate shortcomings of the PoC model. Thereafter, in Chapter 5, an improved model is presented along with an algorithm for its implementation in PySPH. The report concludes with avenues that can be explored to carry out robust simulations for FSI problems with PySPH.